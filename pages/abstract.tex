\chapter{\abstractname}

%TODO: Abstract
% Abstract zu lang: um 2 Abschnitte kürzen, dafür Introduction länger?
Enterprise Architecture Management (EAM) has emerged to be a settled instrument to increase data quality, reduce IT costs and to reduce the error-prone effects of the process of the Enterprise Architecture information collection.

Although EAM has gained importance over more than a decade to improve the alignment of business with IT and transmit a holistic view of the entire organization along with its application landscape, this discipline has been shown to be more complex than expected.

One of the major challenges for an organization is the documentation of Enterprise Architecture (EA) information. Most of the EA documentation is still collected manually. The high number of applications within the application landscape coupled with data redundancy and the inconsistent data leads to a high complexity of EA documentation. This produces a time consuming and highly error-rated documentation and collection process of EA information. Maintaining and gathering information for the EA is (as well) a very expensive task.  The lack of governance is also a major challenge for EA itself. The absence of name conventions, standardized tools integration, continuous delivery and build pipelines, etc., adds to the complexity of a clear and transparent EA documentation.

Since one of the goals of EAM is to create a completely holistic view of the EA, it should integrate external sources, such as cloud services and other repositories.

This thesis will focus on the automated documentation of cloud applications information from an application development pipeline, and the business domain assignments based on the hypothesis that the automated documentation of EA cloud applications leads to a reduction of IT costs and effort, while increasing the data quality of EA information and data.

%TODO: Change. include static vs dynamic data to enrich data of ea tools. dynamic data for monitoring
In order to enhance a holistic view, the EA Tool can be enriched with dynamic data to enable a continuous process of monitoring application performance and infrastructure since the already available documented EA information is mostly static information and data coming from EA data sources.

\textbf{Keywords}: Enterprise Architecture Management, automated documentation, Business Domains, application development pipeline, cloud application information
