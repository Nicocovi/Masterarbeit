\chapter{Appendix} \label{chapter:appendix}

%%%%%%%%%%%%%%%%%%%%%%%%%%%%%%%%%%%%%%%%%%%%%%%%%%%%%%%%%%%%
%% Beginning of questionnaire command definitions %%
%%%%%%%%%%%%%%%%%%%%%%%%%%%%%%%%%%%%%%%%%%%%%%%%%%%%%%%%%%%%

%% \Qq = Questionnaire question. Oh, this is just too simple. It helps
%% making it easy to globally change the appearance of questions.
\newcommand{\Qq}[1]{\textbf{#1}}

%% \QO = Circle or box to be ticked. Used both by direct call and by
%% \Qrating and \Qlist.
\newcommand{\QO}{$\Box$}% or: $\ocircle$

%% \Qrating = Automatically create a rating scale with NUM steps, like
%% this: 0--0--0--0--0.
\newcounter{qr}
\newcommand{\Qrating}[1]{\QO\forloop{qr}{1}{\value{qr} < #1}{---\QO}}

%% \Qline = Again, this is very simple. It helps setting the line
%% thickness globally. Used both by direct call and by \Qlines.
\newcommand{\Qline}[1]{\noindent\rule{#1}{0.6pt}}

%% \Qlines = Insert NUM lines with width=\linewith. You can change the
%% \vskip value to adjust the spacing.
\newcounter{ql}
\newcommand{\Qlines}[1]{\forloop{ql}{0}{\value{ql}<#1}{\vskip0em\Qline{\linewidth}}}

%% \Qlist = This is an environment very similar to itemize but with
%% \QO in front of each list item. Useful for classical multiple
%% choice. Change leftmargin and topsep accourding to your taste.
\newenvironment{Qlist}{%
\renewcommand{\labelitemi}{\QO}
\begin{itemize}[leftmargin=1.5em,topsep=-.5em]
}{%
\end{itemize}
}

%% \Qtab = A "tabulator simulation". The first argument is the
%% distance from the left margin. The second argument is content which
%% is indented within the current row.
\newlength{\qt}
\newcommand{\Qtab}[2]{
\setlength{\qt}{\linewidth}
\addtolength{\qt}{-#1}
\hfill\parbox[t]{\qt}{\raggedright #2}
}

%% \Qitem = Item with automatic numbering. The first optional argument
%% can be used to create sub-items like 2a, 2b, 2c, ... The item
%% number is increased if the first argument is omitted or equals 'a'.
%% You will have to adjust this if you prefer a different numbering
%% scheme. Adjust topsep and leftmargin as needed.
\newcounter{itemnummer}
\newcommand{\Qitem}[2][]{% #1 optional, #2 notwendig
\ifthenelse{\equal{#1}{}}{\stepcounter{itemnummer}}{}
\ifthenelse{\equal{#1}{a}}{\stepcounter{itemnummer}}{}
\begin{enumerate}[topsep=2pt,leftmargin=2.8em]
\item[\textbf{\arabic{itemnummer}#1.}] #2
\end{enumerate}
}

%% \QItem = Like \Qitem but with alternating background color. This
%% might be error prone as I hard-coded some lengths (-5.25pt and
%% -3pt)! I do not yet understand why I need them.
\definecolor{bgodd}{rgb}{0.8,0.8,0.8}
\definecolor{bgeven}{rgb}{0.9,0.9,0.9}
\newcounter{itemoddeven}
\newlength{\gb}
\newcommand{\QItem}[2][]{% #1 optional, #2 notwendig
\setlength{\gb}{\linewidth}
\addtolength{\gb}{-5.25pt}
\ifthenelse{\equal{\value{itemoddeven}}{0}}{%
\noindent\colorbox{bgeven}{\hskip-3pt\begin{minipage}{\gb}\Qitem[#1]{#2}\end{minipage}}%
\stepcounter{itemoddeven}%
}{%
\noindent\colorbox{bgodd}{\hskip-3pt\begin{minipage}{\gb}\Qitem[#1]{#2}\end{minipage}}%
\setcounter{itemoddeven}{0}%
}
}

%%%%%%%%%%%%%%%%%%%%%%%%%%%%%%%%%%%%%%%%%%%%%%%%%%%%%%%%%%%%
%% End of questionnaire command definitions %%
%%%%%%%%%%%%%%%%%%%%%%%%%%%%%%%%%%%%%%%%%%%%%%%%%%%%%%%%%%%%

\section{Evaluation questionnaire}

\subsubsection{1. General data}

\Qq{Company}: \Qline{4cm} \Qq{Date and time}: \Qline{4cm} \\
\Qq{Interviewer}: \Qline{4cm} \Qq{Questionnaire-Nr.}: \Qline{4cm} \\
\Qq{Years of experience}: \Qline{4cm} \Qq{Industry sector}: \Qline{4cm}

\Qitem{\Qq{Which of the below roles are applicable for you?}\\
\QO Enterprise Architect  \QO Product Owner  \QO DevOps  \QO Other: \Qline{3cm}
}
\Qitem{\Qq{It is a lot of effort to document the IT landscape of your company}\\
\QO Strongly agree  \QO agree  \QO disagree  \QO Strongly disagree
}

\Qitem{\Qq{The information in the EA tool is outdated}\\
\QO Yes  \QO No
}
\Qitem{\Qq{Does your company have an automated process for documenting the EA?}\\
\QO Yes  \QO No
}
\Qitem{\Qq{If your company has an automated process for documenting the EA, please describe the process:}
\Qlines{2}
}

\Qitem{\Qq{Does your company run applications/services in a cloud-based environment?}\\
\QO Yes  \QO No
}

\Qitem{\Qq{Does your company consider migrating legacy systems into a cloud-based environment?}\\
\QO Yes  \QO No
}

\subsubsection{2. Agile methodologies}

\Qitem{\Qq{Does your company use a Continuous Delivery approach?}\\
\QO Yes  \QO No
}

\Qitem{\Qq{Who collects the EA information of a new developed application/service for the EA Tool?}\\
\QO Enterprise Architect  \QO Product Owner  \QO DevOps  \QO Other: \Qline{3cm}
}

\Qitem{\Qq{Do you use any of the above mentioned information sources to retrieve EA relevant information?}\\
\QO Network scanners and monitors  \QO CMDB  \QO PPM  \QO ESB  \QO Other: \Qline{2cm}
}

\Qitem{\Qq{Do you use any of the above mentioned information sources to retrieve EA relevant information?}\\
\QO Runtime environment  \QO Jira  \QO Github  \QO Jenkins  \QO Wiki
}

\Qitem{\Qq{Do you think this information sources could contain relevant EA information? Why?}
\Qlines{2}
}

\Qitem{\Qq{How do you collect this information?}
\Qlines{2}
}

\Qitem{\Qq{Does your company have an established toolchain for the development pipeline?}\\
\QO Yes  \QO No
}

\Qitem{\Qq{Do you think an established toolchain improve the process of EA documentation?Why?}\\
\QO Yes  \QO No
\Qlines{2}
}


\Qitem{\Qq{Imposing the team to incorporate a pipeline-script in the repository is easy to establish}\\
\QO Strongly agree  \QO agree  \QO disagree  \QO Strongly disagree
}

\Qitem{\Qq{Imposing the team to use a predefined toolchain for the application development is easy to establish}\\
\QO Strongly agree  \QO agree  \QO disagree  \QO Strongly disagree
}

\Qitem{\Qq{How do you determine an application owner?}\\
\Qlines{2}
}

\Qitem{\Qq{Do you follow any guidelines when developing a new product?}\\
\Qlines{2}
}

\subsubsection{3. Cloud environments}

\Qitem{\Qq{Does your company use any technologies for automating the EA documentation process?}\\
\QO Yes  \QO No
}

\Qitem{\Qq{How does your company collect the applications running in a cloud-based environment?}\\
\QO Yes  \QO No
}

\Qitem{\Qq{Does your company have any architecture guidelines for cloud applications?}\\
\QO Yes  \QO No
}

\Qitem{\Qq{Are you familiar with the 12-factor-app criteria for determining if an application/product is cloud-ready?}\\
\QO Yes  \QO No
}

\Qitem{\Qq{Do you use any technologies for monitoring applications?}\\
\QO Yes  \QO No
}

\Qitem{\Qq{If yes, please give a short answer regarding the technologies used for monitoring the applications/services:}\\
\Qlines{2}
}

\subsubsection{4. Tool Evaluation}

\Qitem{\Qq{The presented prototype covers different views for the different stakeholders such as product owners, enterprise architects and devops-teams}\\
\QO Strongly agree  \QO agree  \QO disagree  \QO Strongly disagree
}

\Qitem{\Qq{The presented prototype enables a federated approach linking the tools used during an application development pipeline}\\
\QO Strongly agree  \QO agree  \QO disagree  \QO Strongly disagree
}

\Qitem{\Qq{The following information displayed in the general section of the detailed view of an application/service is useful}\\
\QO Status  \QO Name  \QO Description  \QO Domain  \QO Subdomain \\
\QO Product  \QO Owner  \QO Json link  \QO Changes  \QO Additional information
}

\Qitem{\Qq{The following information displayed in the general section of the detailed view of an application/service is NOT useful.Why?}\\
\QO Status  \QO Name  \QO Description  \QO Domain  \QO Subdomain \\
\QO Product  \QO Owner  \QO Json link  \QO Changes  \QO Additional information
\Qlines{2}
}

\Qitem{\Qq{I would like to see the following information displayed in the general section:}
\Qlines{2}
}

\Qitem{\Qq{The following information in the runtime section is useful}\\
\QO Instances  \QO Ram  \QO CPU  \QO Disk  \QO Host
}

\Qitem{\Qq{The following information in the runtime section is NOT useful. Why?}\\
\QO Instances  \QO Ram  \QO CPU  \QO Disk  \QO Host
\Qlines{2}
}

\Qitem{\Qq{I would like to see the following information displayed in the runtime section:}
\Qlines{2}
}

\Qitem{\Qq{The following information displayed is the metrics section is useful}\\
\QO URL  \QO Prometheus metrics endpoint  \QO Response time
}

\Qitem{\Qq{The following information displayed is the metrics section is NOT useful. Why?}\\
\QO URL  \QO Prometheus metrics endpoint  \QO Response time
\Qlines{2}
}

\Qitem{\Qq{I would like to see the following information displayed in the metrics section:}
\Qlines{2}
}

\Qitem{\Qq{The information displayed in the Application/Services section is useful.If no, why?}\\
\QO Yes  \QO No
\Qlines{2}
}

\Qitem{\Qq{The information displayed in the software dependencies section is useful. Why?}\\
\QO Yes  \QO No
\Qlines{2}
}

\Qitem{\Qq{The following information in the Jira monitoring section is useful:}\\
\QO Total issues  \QO Open issues  \QO Components  \QO Project progress
}

\Qitem{\Qq{The following information in the Jira monitoring section is NOT useful. Why?}\\
\QO Total issues  \QO Open issues  \QO Components  \QO Project progress
\Qlines{2}
}

\Qitem{\Qq{I would like to see the following information displayed in the jira  section:}
\Qlines{2}
}

\Qitem{\Qq{The following information in the github monitoring section is useful:}\\
\QO Contributors  \QO Lines of Code  \QO Commit activity
}

\Qitem{\Qq{The following information in the github monitoring section is NOT useful. Why?}\\
\QO Contributors  \QO Lines of Code  \QO Commit activity
\Qlines{2}
}

\Qitem{\Qq{What information would you like to see in the Github monitoring section regarding the repository of the application?}
\Qlines{2}
}

\Qitem{\Qq{The following information displayed in the Jenkins job monitoring section is useful:}\\
\QO Build number  \QO Duration  \QO Estimated duration  \QO Result  \QO Timestamp
}

\Qitem{\Qq{The following information displayed in the Jenkins job monitoring section is NOT useful. Why?}\\
\QO Build number  \QO Duration  \QO Estimated duration  \QO Result  \QO Timestamp
\Qlines{2}
}

\Qitem{\Qq{An automated test 12-factor evaluation of the application/service is helpful to evaluate if the application/service is cloud-ready:}\\
\QO Yes  \QO No
}

\Qitem{\Qq{An automated test of the resilience pattern is helpful to determine the elasticity of an application/service:}\\
\QO Yes  \QO No
}

\Qitem{\Qq{An integration of the architecture belt tool would improve the governance monitoring section:}\\
\QO Yes  \QO No
}

\Qitem{\Qq{The presented sections represent relevant EA information sources}\\
\QO Yes  \QO No
}

\Qitem{\Qq{Would you include other information sources?}\\
\QO Yes  \QO No
} 

\Qitem{\Qq{If yes please describe briefly why:}\\
\Qlines{2}
} 

\Qitem{\Qq{The visualizations displayed in the visualizations tab is useful:}\\
\QO Yes  \QO No
} 

\Qitem{\Qq{I would like to see the following visualizations:}
\Qlines{2}
}

\Qitem{\Qq{Other comments:}
\Qlines{3}
}

\subsubsection{5. Overall}

\Qitem{\Qq{The presented process will automate the EA documentation of applications/services running on a cloud-based environment}\\
\QO Strongly agree  \QO agree  \QO disagree  \QO Strongly disagree
}

\Qitem{\Qq{Overall score of the tool: Being 5 the best}\\
\QO 1  \QO 2  \QO 3  \QO 4  \QO 5
}